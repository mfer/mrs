\documentclass[conference]{IEEEtran}
\hyphenation{op-tical net-works}
\begin{document}
  \title{Introduction to Approximation Algorithms - 2014/2 \\ Practical Project Proposal}
  \author{
    \IEEEauthorblockN{Flavio Haueisen {\it and} Manasses Ferreira}
    \IEEEauthorblockA{Computer Science Department UFMG\\ 
      Belo Horizonte, Minas Gerais, Brasil\\
      Email: \{haueisen, mfer\}@dcc.ufmg.br}
  }
  \maketitle
  \begin{abstract}
    The main objective of this Practical Project is to extend and validate 
    the propositions of the Seminar chosen paper. The extension proposed
    is to solve other problem with the approximation algorithm studied and
    the validation will be obtained through simulations on both best and worst case scenarios.
  \end{abstract}
  \section{Introduction}
    The chosen paper \cite{pengjun:2014} solves the Maximum-Weighted Independent Set of Links Problem ({\bf MWIS})
    using algorithms designed under the {\it protocol} interference model. 
    In general, a protocol interference model specifies a pairwise conflict relations among all links.
    A subset is independent if all links are pairwise conflict-free.
    This paper describes several algorithms, we are interested in the orientation one named {\bf OrientWIS}.

    There is a related work on Wireless Communications \cite{olga:2012} in which the authors made a proof (Theorem 4.5) that 
    any algorithm to solve {\bf MWIS} is a constant approximation for the Multi-Rate Scheduling Problem ({\bf MRS}).

    So, our proposed contributions are:
    \begin{enumerate} 
      \item obtains a constant approximation algorithm to {\bf MRS}
      \item implements {\bf OrientWIS}
      \item simulates both best and worst case scenarios
    \end{enumerate}
  \section{Practical Project Tasks}
    This project is divided in two parts: analytical and experimental.
  \subsection{Analytical}
    Into this part, 
    the main task will be map the {\bf MRS} \cite{olga:2012} 
    into a kind of input to the {{\bf OrientWIS} described at \cite{pengjun:2014}.
    This map must preserve the approximation guarantees.
  \subsection{Experimental}
    Simulations will be used to validate {\bf OrientWIS} in best and worst case scenarios.
    The instances used by simulations for best and worst case scenarios will be manually crafted
    based on the nature of the {\bf OrientWIS} algorithm.
    The implementation code will be available at github \cite{mfer:gitMRS}.
  \newpage
  \section{Equations}

  \begin{equation}
    \rho = 2(1+\epsilon)\alpha^{in}_{D}= C\left(1+\frac{2a}{b}\right)^{2} 
    \Longrightarrow 
    \epsilon = \frac{C\left(1+\frac{2a}{b}\right)^{2}}{2\alpha^{in}_{D}} - 1
  \end{equation}    

  \begin{equation}
    \begin{array}{rl}
      \alpha^{in}_{D}= \left\lceil \frac{\pi}{\arcsin \left( \frac{c-1}{2c} \right)} \right\rceil\\
      c = z_{min}
    \end{array}     
    \left\}
      \alpha^{in}_{D}= \left\lceil \frac{\pi}{\arcsin \left( \frac{z_{min}-1}{2z_{min}} \right)} \right\rceil 
    \right.
  \end{equation}  

  $\epsilon = \frac{C}{2\pi}\left(1+\frac{2a}{b}\right)^{2} \arcsin \left( \frac{z_{min}-1}{2z_{min}} \right)  - 1$

  \begin{equation}
    2(1+\epsilon)\alpha^{in}_{D}= C\left(1+\frac{2a}{b}\right)^{2} 
  \end{equation}        

  \begin{equation}
    \begin{array}{rl}
      \rho = 2(1+\epsilon)\alpha^{in}_{D}= C\left(1+\frac{2a}{b}\right)^{2} 
      \Longrightarrow 
      \epsilon = \frac{C\left(1+\frac{2a}{b}\right)^{2}}{2\alpha^{in}_{D}} - 1 \\
      \begin{array}{rl}
        \alpha^{in}_{D}= \left\lceil \frac{\pi}{\arcsin \left( \frac{c-1}{2c} \right)} \right\rceil\\
        c = z_{min}
      \end{array}     
      \left\}
        \alpha^{in}_{D}= \left\lceil \frac{\pi}{\arcsin \left( \frac{z_{min}-1}{2z_{min}} \right)} \right\rceil 
      \right.
    \end{array}
    \left\}
      \epsilon = \frac{C}{2\pi}\left(1+\frac{2a}{b}\right)^{2} \arcsin \left( \frac{z_{min}-1}{2z_{min}} \right)  - 1
    \right.  
  \end{equation}  

  \begin{equation}
    \epsilon(C,a,b,z_{min})
  \end{equation}        


$
      a \leq 2 \Delta_{l} d_{min} z_{min} 
      \left \{  
        \frac{2 \sqrt{3}\pi \alpha(\alpha-1)}{3(\alpha-2)}
        \frac{\Delta_{\beta}\beta_{min}(\Delta_{l}d_{min})^{\alpha}}{(d_{min} z_{min})^\alpha}
      \right\}^{\frac{1}{\alpha-2}}
$

$$
      \leq 2 \Delta_{l} 
      \left \{  
        \frac{2 \sqrt{3}\pi \alpha(\alpha-1)}{3(\alpha-2)}
        \Delta_{l}^{\alpha}\Delta_{\beta}
        \frac{\beta_{min}(d_{min})^{\alpha}(d_{min} z_{min})^{\alpha-2}}{(d_{min} z_{min})^\alpha}
      \right\}^{\frac{1}{\alpha-2}}
$$


$
\frac{(d_{min})^{\alpha}(d_{min} z_{min})^{\alpha-2}}{(d_{min} z_{min})^\alpha}
=\frac{d_{min}^{\alpha-2}}{z_{min}^2}
=d_{min}^{\alpha-2}\left(\frac{(\alpha-2)6}{\beta_{min}\alpha 4\pi\sqrt{3}}\right)^\frac{2}{\alpha}
$

  $$
      \leq 2 \Delta_{l} 
      \left \{  
        (\alpha-1)\frac{2 \sqrt{3}\pi \alpha}{3(\alpha-2)}
        \Delta_{l}^{\alpha}\Delta_{\beta}d_{min}^{\alpha-2}\beta_{min}^\frac{\alpha-2}{\alpha}
        \left(\frac{(\alpha-2)3}{\alpha 2\pi\sqrt{3}}\right)^\frac{2}{\alpha}
      \right\}^{\frac{1}{\alpha-2}}
  $$
  $$
      \leq 2 \Delta_{l} 
      \left \{  
        \left( \frac{2 \sqrt{3}\pi \alpha}{3(\alpha-2)} \right)^\frac{\alpha-2}{\alpha}
        \Delta_{l}^{\alpha}\Delta_{\beta}d_{min}^{\alpha-2}\beta_{min}^\frac{\alpha-2}{\alpha}        
      \right\}^{\frac{1}{\alpha-2}}
  $$

  \begin{equation}
    \epsilon = \frac{C}{2\pi}\left(1+\frac{2a}{b}\right)^{2} \arcsin \left( \frac{z_{min}-1}{2z_{min}} \right)  - 1
  \end{equation}

  \begin{equation}
      C = \frac{\sqrt{3}\pi}{6} \qquad \qquad
      a \leq 2 \Delta_{l}^{\frac{2(\alpha-1)}{\alpha-2}}
      \Delta_{\beta}^{\frac{1}{\alpha-2}}
      \left( \frac{2 \sqrt{3}\pi \alpha}{3(\alpha-2)} \right)^\frac{1}{\alpha}
      d_{min}
      \beta_{min}^\frac{1}{\alpha} \qquad \qquad
      b = \beta_{min}^{\frac{1}{\alpha}}-1
  \end{equation}

  \begin{equation}
      \frac{2a}{b} \leq 
        4 \Delta_{l}^{\frac{2(\alpha-1)}{\alpha-2}}
        \Delta_{\beta}^{\frac{1}{\alpha-2}}
        \left( \frac{2 \sqrt{3}\pi \alpha}{3(\alpha-2)} \right)^\frac{1}{\alpha}
        d_{min}
        \left(1-\frac{1}{\beta_{min}^\frac{1}{\alpha}}\right)^{-1}
  \end{equation}

  \begin{equation}
    \epsilon(\Delta_{l},\Delta_{\beta},d_{min},\beta_{min},\alpha)
  \end{equation}    

  \begin{equation}
    \epsilon \leq \frac{\sqrt{3}}{12}
    \left[ 1 +  
      4 \Delta_{l}^{\frac{2(\alpha-1)}{\alpha-2}}
      \Delta_{\beta}^{\frac{1}{\alpha-2}}
      \left( \frac{2 \sqrt{3}\pi \alpha}{3(\alpha-2)} \right)^\frac{1}{\alpha}
      d_{min}
      \left(1-\frac{1}{\beta_{min}^\frac{1}{\alpha}}\right)^{-1}    
    \right]^{2}
    \arcsin \left(
      \frac{1}{2}-
      \left(\frac{3(\alpha-2)}{\alpha 2\pi\sqrt{3}} \right)^\frac{1}{\alpha}
    \right)  
  \end{equation}    

  \begin{thebibliography}{1}
    \bibitem{olga:2012}
      Olga Goussevskaia et al, 
      \emph{Wireless Multi-Rate Scheduling: From Physical Interference to Disk Graphs}, 
      3rd~ed.\hskip 1em plus 0.5em minus 0.4em\relax 
      Harlow, England: Addison-Wesley, 1999.
    \bibitem{pengjun:2014}
      Peng-Jun Wan et al, 
      \emph{Fast and Simple Approximation Algorithms for Maximum Weighted Independet Set of Links}, 
      IEEE INFOCOM 2014 - \hskip 1em plus 0.5em minus 0.4em\relax 
      IEEE Conference on Computer Communications.
    \bibitem{mfer:gitMRS}
      Flavio Haueisen {\it and} Manasses Ferreira,
      \emph{Multi-Rate Scheduling source-code},
      GitHub Repository,
      https://github.com/mfer/mrs.git
  \end{thebibliography}
\end{document}